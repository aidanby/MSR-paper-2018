\section{Motivating Example}
We study the \textit{Trivial-Graph} \cite{akollegger_2011} GitHub project to demonstrate the importance of maintaining the traceability between \textit{.feature} files and source code files. \textit{Trivial-Graph} is a web-based trivia game application in which teams compete to answer questions in multiple rounds. \textit{Trivial-Graph} uses the \textit{Cucumber} BDD framework \cite{akollegger_2011, BDDFrame} to write \textit{.feature} files and \textit{step definition} files. The \textit{.feature} files describe the scenarios (i.e., functionalities) of the project. The \textit{step definition} files are automatically generated by \textit{Cucumber} based on the \textit{.feature} files. The \textit{step definition} files define the tests for the functionalities specified in the \textit{.feature} files. 
Figure 1 shows an example of a \textit{.feature} file and its corresponding \textit{step definition} file. A \textit{step definition} file has annotations (e.g., ``@When" and ``@Then") to indicate which step of a \textit{.feature} file is tested by a given test method. The developer then runs the tests on the source code using the \textit{step definition} files, until every tests passes, satisfying the functions of each scenario

\begin{figure}
	\textbf{\textit{.feature} file code}:
	\begin{lstlisting}
	@players
	Feature: Trivialt Players and Teams
	Scenario: Register as a new player
	When you register "Tobias" with handle "@thobe"
	Then trivialt knows "@thobe" is "Tobias"
	And "@thobe" should be the current player
	\end{lstlisting}
	\textbf{Step definition file code}:
	\begin{lstlisting}
	@players
	@When("^you register \"([^\"]*)\" with handle \"([^\"]*)\"$")
	public void youRegisterUser(String name, String handle){
		currentPlayer = trivialtWorld.register(handle, name);
		assertThat(currentPlayer, is(not(nullValue())));
	}
	@Then("^trivialt knows \"([^\"]*)\" is \"([^\"]*)\"$")
	public void trivialtKnowsPlayer(String handle, String name){
	        Player foundPlayer = trivialtWorld.findPlayer(handle);
			assertThat(foundPlayer, is(not(nullValue())));
			assertThat(foundPlayer.getName(), is(name));
	}
	 @Then("^\"([^\"]*)\" should be the current player$")
	public void assertTheCurrentPlayerIs(String handle){
		assertThat(currentPlayer.getHandle(), is(handle));
	}
	\end{lstlisting}
	\caption{Code example from BDD files}
	
\end{figure}

\begin{figure*}
	\centering
	\includegraphics[height=4.7in, width=6.5in]{timeline.pdf}
	\caption{Time line of an example BDD project}
\end{figure*} 
We observe co-changes between \textit{.feature} files and source code files in the \textit{Trivial-Graph} by studying six commits in a one month time frame during 2011 (shown in Figure 2). Figure 2 shows an out-of-sync co-change in the dotted lines surrounding commits 4 and 5. In commit 1, the author creates the project. A week later, the author adds the \textit{app.feature} file in commit 2 as well as the corresponding source code files. Similarly, in commit 3, the author adds the \textit{team\_players.feature} file and the corresponding source code files (i.e., Add \textit{Teams.java} and \textit{Players.java}). The author then follows the requirements described by \textit{app.feature} and \textit{team\_players.feature} to add LOC in App.java and \textit{Players.java} in commit 4. However, in commit 4, the author does not modify \textit{.feature} files to describe the newly added source code functionalities. Roughly two hours later, in commit 5, the author adds LOC to \textit{app.feature} and \textit{team\_players.feature} and writes the following commit message ``Added missing steps". In each of these commits, we observe co-changing \textit{.feature} files and source code files (i.e., \textit{app.feature} co-changing with \textit{App.java}, and team\_players.feature co-changing with \textit{Players.java}). However, the co-changing files are not modified simultaneously. We identify \textit{.feature} file co-changes across different commits, i.e., source code files in commit 4 co-changing with \textit{.feature} files in commit 5. We observe that, although\textit{ app.feature} and \textit{AppTest.java} are co-changing files, the author does not include \textit{app.feature} changes in commit 4, showing a temporary \textit{out-of-sync} co-change. As a result of this out-of-sync co-change, the author needs to make another commit roughly two hours after commit 4. After step 6, we observe that \textit{app.feature} is modified with \textit{AppTest.java} in the same commit frequently, showing a continuing co-change link between the two files.

As the project scales up, it becomes harder to understand which \textit{.feature} file corresponds with which source file or methods, and out-of-sync co-changes become more prevalent. In commit 5 of Figure 2, we observe that \textit{team\_players.feature} correspond to three different source files, and the \textit{.feature} file is still modified 18 days after its first co-changing source file was added. As the project grows larger, the \textit{.feature} file will need to be modified to describe functionalities of other source files, causing the \textit{.feature} file maintenance time even longer. For a developer joining such a project, it is time-consuming and challenging to identify every co-change between \textit{.feature} files and source code files. We explore approaches that can ease the traceability of co-changes between \textit{.feature} files and source code files and reduce the out-of-sync co-changes. 
