\section{Conclusions}
BDD is a relatively new testing strategy that uses English-like syntax in describing code functionalities. BDD makes it easier for all stakeholders involved in a project to understand the functionalities of a software project. With the use of \textit{.feature} files in describing the functionalities of a software, the co-evolution of \textit{.feature} files and source code files must be kept up-to-date. In our work, we detect the co-changes between \textit{.feature} files and source code files, and find characteristics that can accurately predict the co-changes in order to improve the traceability between \textit{.feature} files and source code files.

Our approach can link \textit{.feature} files with source code files with an accuracy of 79\% and predict co-changes between \textit{.feature} files and source code files with an AUC of 0.77. \textit{Test files added}, \textit{other files modified}, \textit{test files renamed}, and \textit{source LOC deleted} are the best predictors for co-changes between \textit{.feature} files and source code files. Our results demonstrate that co-changes between \textit{.feature} files and source code files can be detected, and that source code change characteristics can predict those co-changes. Our findings can help developers to keep software documentation (i.e., \textit{.feature} files) up-to-date and help projects to adopt the BDD practices in developing software more efficiently. To further assist BDD developers, we plan to explore BDD co-changes with regards to more languages and analyze more projects using BDD.

